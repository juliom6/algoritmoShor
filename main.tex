\documentclass[12pt]{article}
\usepackage[spanish]{babel} % espanol, ingles
\usepackage[utf8]{inputenc}
%\usepackage[round,sort,nonamebreak]{natbib}

\title{El algoritmo de Shor}
\author{Varios\\ {\tt varios@gmail.com}}
\date{}
\usepackage{ulem}
\usepackage{cancel}
\usepackage{natbib}
\usepackage{graphicx}
\usepackage{fourier}
\usepackage{ragged2e}
\usepackage{amsthm}

\newtheorem{teorema}{Teorema}
\theoremstyle{definition}
\newtheorem{definicion}{Definici\'on}

\begin{document}
\maketitle

\section{Introducción}
Segun \citet{Shor:1994},
\section{Una revisión de mecánica cuántica para computación cuántica}
\section{Circuitos cuánticos}
\section{La factorización puede ser reducida al cálculo del orden}
\section{El algoritmo cuántico para el cálculo del orden}
\section{La transformada discreta de Fourier}
\section{Generalización por medio de un ejemplo}
\section{La transformada de Fourier en terminos de compuertas universales}

\bibliographystyle{plainnat-ime}
\bibliography{references}
\end{document}
